\documentclass[11pt]{article}

\usepackage{amssymb}
\usepackage{amsfonts}
\usepackage{amsmath}
\usepackage{mathtools}
\usepackage{amsthm}

\usepackage[letterpaper,margin=1in]{geometry}

\usepackage{enumerate}

      \theoremstyle{plain}
      \newtheorem{theorem}{Theorem}
      \newtheorem{lemma}[theorem]{Lemma}
      \newtheorem{corollary}[theorem]{Corollary}
      \newtheorem{proposition}[theorem]{Proposition}
      \newtheorem{conjecture}[theorem]{Conjecture}
      \newtheorem{question}[theorem]{Question}
      \newtheorem{claim}[theorem]{Claim}

      \theoremstyle{definition}
      \newtheorem{definition}[theorem]{Definition}
      \newtheorem{notation}[theorem]{Notation}
      \newtheorem{example}[theorem]{Example}
      \newtheorem{observation}[theorem]{Observation}
      \newtheorem{game}[theorem]{Game}

      \theoremstyle{remark}
      \newtheorem{remark}[theorem]{Remark}

      \theoremstyle{plain}
      \newtheorem*{theorem*}{Theorem}
      \newtheorem*{lemma*}{Lemma}
      \newtheorem*{corollary*}{Corollary}
      \newtheorem*{proposition*}{Proposition}
      \newtheorem*{conjecture*}{Conjecture}
      \newtheorem*{question*}{Question}
      \newtheorem*{claim*}{Claim}

      \theoremstyle{definition}
      \newtheorem*{definition*}{Definition}
      \newtheorem*{example*}{Example}
      \newtheorem*{observation*}{Observation}
      \newtheorem*{game*}{Game}

      \theoremstyle{remark}
      \newtheorem*{remark*}{Remark}

\title{Limited Information Strategies and Discrete Selectivity}
\author{Jared Holshouser and Steven Clontz}

\usepackage{clontzDefinitions}

\newcommand{\bmPoGame}[2]{BM_{po}(#1,#2)}

\begin{document}

\maketitle

\section{Definitions}

\begin{definition}
  The \term{selection principle} \(\schSelProp{\mc A}{\mc B}\) states that
  given \(A_n\in\mc A\) for \(n<\omega\), there exist \(B_n\in[A_n]^{<\omega}\)
  such that \(\bigcup_{n<\omega}B_n\in\mc B\).
\end{definition}

\begin{definition}
  The \term{selection game} \(\schSelGame{\mc A}{\mc B}\) is the
  analogous game to \(\schSelProp{\mc A}{\mc B}\), where during each
  round \(n<\omega\), Player \(\plI\) first
  chooses \(A_n\in\mc A\), and then Player \(\plII\) chooses
  \(B_n\in[A_n]^{<\omega}\).
  Player \(\plII\) wins in the case that \(\bigcup_{n<\omega}B_n\in\mc B\),
  and Player \(\plI\) wins otherwise.
\end{definition}

\begin{definition}
  Let \(\mc O_X\) be the collection of open covers for a topological space
  \(X\). Then \(\schSelProp{\mc O_X}{\mc O_X}\) is the well-known
  \term{Menger property} for \(X\) (\(M\) for short), and
  \(\schSelGame{\mc O_X}{\mc O_X}\) is the
  well-known \term{Menger game}.
\end{definition}

\begin{definition}
  An \term{\(\omega\)-cover} \(\mc U\)
  for a topological space \(X\) is an open cover
  such that for every \(F\in[X]^{<\omega}\), there exists some \(U\in\mc U\)
  such that \(F\subseteq U\).
\end{definition}

\begin{definition}
  Let \(\Omega_X\) be the collection of \(\omega\)-covers for a topological
  space \(X\). Then \(\schSelProp{\Omega_X}{\Omega_X}\) is the
  \term{\(\Omega\)-Menger property} for \(X\) (\(\Omega M\) for short), and
  \(\schSelGame{\Omega_X}{\Omega_X}\) is the \term{\(\Omega\)-Menger game}.
\end{definition}

\begin{definition}
  Let \(\Omega_{X,x}\) be the collection of subsets \(A\subset X\) where
  \(x\in\closure{A}\). (Call \(A\) a \term{blade} of \(x\).)
  Then \(\schSelProp{\Omega_{X,x}}{\Omega_{X,x}}\) is the
  \term{countable fan tightness property} for \(X\) at \(x\)
  (\(CFT_x\) for short), and
  \(\schSelGame{\Omega_{X,x}}{\Omega_{X,x}}\) is the
  \term{countable fan tightness game} for \(X\) at \(x\).
\end{definition}

\begin{definition}
  A space \(X\) has \term{countable fan tightness} (\(CFT\) for short)
  if it has
  countable fan tightness at each point \(x\in X\).
\end{definition}

\begin{definition}
  Let \(\mc D_{X}\) be the collection of dense subsets of a topological
  space \(X\).
  Then \(\schSelProp{\mc D_X}{\Omega_{X,x}}\) is the
  \term{countable dense fan tightness property} for \(X\) at \(x\)
  (\(CDFT_x\) for short), and
  \(\schSelGame{\mc D_X}{\Omega_{X,x}}\) is the
  \term{countable dense fan tightness game} for \(X\) at \(x\).
\end{definition}

\begin{definition}
  A space \(X\) has \term{countable dense fan tightness}
  (\(CDFT\) for short) if it has
  countable dense fan tightness at each point \(x\in X\).
\end{definition}

\begin{definition}
  \(\schSelProp{\mc D_X}{\mc D_X}\) is the
  \term{selective separability property} for \(X\)
  (\(SS\) for short), and
  \(\schSelGame{\mc D_X}{\mc D_X}\) is the
  \term{selective separability game} for \(X\).
\end{definition}

\begin{definition}
  A \term{strategy} for \(\plII\) in the game \(\schSelGame{\mc A}{\mc B}\)
  is a function \(\sigma\) satisfying
  \(\sigma(\<A_0,\dots,A_n\>)\in[A_n]^{<\omega}\) for
  \(\<A_0\,\dots,A_n\>\in\mc A^{n+1}\). We say this strategy is
  \term{winning} if whenever \(\plI\) plays \(A_n\in\mc A\) during each
  round \(n<\omega\), \(\plII\) wins the game by playing
  \(\sigma(\<A_0,\dots,A_n\>)\) during each round \(n<\omega\).
  If a winning strategy exists, then we write
  \(\plII\win\schSelGame{\mc A}{\mc B}\).
\end{definition}

\begin{definition}
  A \term{Markov strategy} for \(\plII\) in the game
  \(\schSelGame{\mc A}{\mc B}\)
  is a function \(\sigma\) satisfying
  \(\sigma(A,n)\in[A_n]^{<\omega}\) for
  \(A\in\mc A\) and \(n<\omega\). We say this Markov strategy is
  \term{winning} if whenever \(\plI\) plays \(A_n\in\mc A\) during each
  round \(n<\omega\), \(\plII\) wins the game by playing
  \(\sigma(A_n,n)\) during each round \(n<\omega\).
  If a winning Markov strategy exists, then we write
  \(\plII\markwin\schSelGame{\mc A}{\mc B}\).
\end{definition}

\begin{definition}
In some instances, player I will be able to win a game regardless of what II is playing.
In this case, it is possible to have a strategy for I which depends only on the round of the game.
We say I has a \term{pre-determined strategy} and write \(\plI\prewin G\).
\end{definition}

\begin{notation}
  If \(\schSelProp{\mc A}{\mc B}\) characterizes the property \(P\),
  then we say \(\plII\win\schSelGame{\mc A}{\mc B}\) characterizes
  \(P^+\) (``strategically \(P\)''), and
  \(\plII\markwin\schSelGame{\mc A}{\mc B}\) characterizes
  \(P^{\plusMark}\) (``Markov \(P\)'').
  Of course, \(P^{\plusMark}\Rightarrow P^+ \Rightarrow P\).
\end{notation}

\begin{notation}
Let \(\schStrongSelProp{\mc A}{\mc B},\schStrongSelGame{\mc A}{\mc B}\)
be the natural variants of
\(\schSelProp{\mc A}{\mc B},\schSelGame{\mc A}{\mc B}\) where each choice
by \(\plII\) must either be a single element or singleton
(whichever is more convenient for the proof at hand), rather than a finite
set. Convention calls for denoting these as
\term{strong} versions of the corresponding selection principles and games,
although the ``strong Menger'' property is commonly known as ``Rothberger''. We will thus
call ``strong \(\Omega\)-Menger'' ``\(\Omega\)-Rothberger'' and shorten it
with \(\Omega R\), and otherwise attach the prefix ``s''
when abbreviating to all other strong variants.
\end{notation}

In addition to pure selection games, we also will be playing various point-picking games.

\begin{definition}
Set \(T(X)\) to be the non-empty open subsets of \(X\).
The \term{point-open game} for \(X\), denoted \(PO(X)\), is played as follows.
Each round, player I plays a point \(x_n \in X\) and player II plays an open sets \(U_n\) with the property that \(x_n \in U_n\).
I wins the play of the game if \(X = \bigcup_n U_n\).

The \term{finite-open game} for \(X\), denoted \(FO(x)\), is played similarly, except that I now plays finite subsets of \(X\), and II's open sets must cover I's finite set.
Note that \(PO(X)\) is just \(sFO(S)\).
\(\Omega FO(X)\) and \(\Omega PO(X)\) are defined according to convention: I now wins if \(\{U_n : n \in \omega\}\) forms an \(\omega\)-cover of \(X\).
\end{definition}

\begin{definition}
Let \(x \in X\).
\term{Gruenhage's \(W\)-game} for \(x\), denoted \(\gruConGame{X}{x}\), is played as follows.
Each round, player I plays an open set \(U_n\) with the property that \(x \in U_n\) and player II plays a point \(x_n \in U_n\).
I wins if \(x_n \to x\).

The \term{closure game} for \(x\), denoted \(CL(X,x)\), is played the same as Gruenhage's \(W\)-game, but now I wins if \(x \in \overline{\{x_n : n \in \omega\}}\).
Note that this is \(\schStrongSelGame{T(X)}{\Omega_{X,x}}\).

The \term{discrete selectivity game}, denoted \(CD(X)\), is also played the same as Gruenhage's \(W\)-game, but now II wins if \(\{x_n : n \in \omega\}\) is closed and discrete.
Note that this is \(\schStrongSelGame{T(X)}{\mathcal{CD}}\) if we let \(\mathcal{CD}\) denoted the closed discrete subsets of \(X\).
\end{definition}

\begin{definition}
A space \(X\) is \term{discretely selective} iff 
\end{definition}

\section{2-marks in CD(X)}

Let 
\(
  [f,F,\epsilon]
    =
  \{g\in C_p(X):|g(x)-f(x)|<\epsilon\text{ for all }x\in F\}
\).

\begin{game}
  Let \(G\) be the following game. During round \(n\), player \(\plI\)
  chooses \(\beta_n<\omega_1\), and player \(\plII\) chooses
  \(F_n\in[\omega_1]^{<\aleph_0}\). \(\plII\) wins if
  whenever \(\gamma<\beta_n\) for co-finitely many \(n<\omega\),
  \(\gamma\in F_n\) for infinitely many \(n<\omega\).
\end{game}

For \(f\in\omega^\alpha\), let 
\(f^{\leftarrow}[n]=\{\beta<\alpha:f(\beta)<n\}\).

\begin{proposition}
  \(\plII \kmarkwin{2} G\).
\end{proposition}
\begin{proof}
  Let \(\{f_\alpha\in\omega^\alpha:\alpha<\omega_1\}\) be a collection
  of pairwise almost-compatible finite-to-one functions.
  Define a \(2\)-mark \(\sigma\) for \(\plII\) by
  \[\sigma(\<\alpha\>,0)=\emptyset\] and
  \[
    \sigma(\<\alpha,\beta\>,n+1)\
      =
    f_\beta^\leftarrow[n]
      \cup
    \{\gamma<\alpha\cap\beta:f_\alpha(\gamma)\not=f_\beta(\gamma)\}
  .\]
  
  Let \(\nu\) be an attack by \(\plI\) against \(\sigma\), and let
  \(\gamma<\nu(n)\) for \(N\leq n<\omega\).
  If \(f_{\nu(n)}(\gamma)\not=f_{\nu(n+1)}(\gamma)\) for infinitely-many
  \(N\leq n<\omega\), then \(\gamma\in\sigma(\<\nu(n),\nu(n+1)\>,n+1)\)
  for infinitely-many \(N\leq n<\omega\).
  Otherwise \(f_{\nu(n)}(\gamma)=f_{\nu(n+1)}(\gamma)=M\) for cofinitely-many
  \(N\leq n<\omega\), so \(\gamma\in\sigma(\<\nu(n),\nu(n+1)\>,n+1)\)
  for cofinitely-many \(N\leq n<\omega\). Therefore \(\sigma\) is a winning \(2\)-mark.
\end{proof}

\begin{theorem}
  \(\plI\kmarkwin{2}CD(C_p(\omega_1+1))\)
\end{theorem}
\begin{proof}
  Let \(\sigma\) be a winning 2-mark for \(\plII\) in \(G\).

  Given a point \(f\in C_p(\omega_1+1)\),
  let \(\alpha_f<\omega_1\) satisfy \(f(\beta)=f(\gamma)\) for all
  \(\alpha_f\leq\beta\leq\gamma\leq\omega_1\).

  Let \(\tau(\emptyset,0)=[\mathbf{0},\{\omega_1\},4]\),
  \(\tau(\<f\>,1)=[\mathbf{0};\sigma(\<\alpha_f\>,0)\cup\{\omega_1\};2]\), and
  \[\tau(\<f,g\>,n+2)=[\mathbf{0};\sigma(\<\alpha_f,\alpha_g\>,n+1)\cup\{\omega_1\};2^{-n}].\]

  Let \(\nu\) be a legal attack by \(\plII\) against \(\sigma\).
  For \(\beta\leq\omega_1\), if \(\beta<\alpha_{\nu(n)}\)
  for co-finitely many \(n<\omega\), then
  \(\beta\in\sigma(\<\alpha_{\nu(n)},\alpha_{\nu(n+1)}\>)\) for
  infinitely-many \(n<\omega\), and thus \(0\in\operatorname{cl}\{\nu(n)(\beta):n<\omega\}\).
  Otherwise \(\beta\geq\alpha_{\nu(n)}\) for infinitely many \(n<\omega\),
  and thus \(0\in\operatorname{cl}\{\nu(n)(\beta):n<\omega\}\) as well.
  Thus \(\mathbf{0}\in\operatorname{cl}\{\nu(n):n<\omega\}\).
\end{proof}

\section{Combining game results}

\begin{theorem}
The following are equivalent for \(T_{3.5}\) spaces \(X\).
\begin{enumerate}[a)]
\item \(X\) is \(R^+\), that is, \(\plII\win \schStrongSelGame{\mc O_X}{\mc O_X}\).
\item \(\plI\win PO(X)\). 
\item \(\plI\win FO(X)\).
\item \(\plI\win\Omega FO(x)\).
\item \(\plI\win \gruConGame{C_p(X)}{\mathbf 0}\).
\item \(\plI\win CL(C_p(X),\mathbf 0)\).
\item \(\plI\win CD(C_p(X))\).
\item \(X\) is \(\Omega R^+\), that is, 
  \(\plII\win \schStrongSelGame{\Omega_X}{\Omega_X}\).
\item \(C_p(X)\) is \(sCFT^+\), that is, 
  \(\plII\win \schStrongSelGame{\Omega_{C_p(X),\mathbf 0}}{\Omega_{C_p(X),\mathbf 0}}\).
\item \(C_p(X)\) is \(sCDFT^+\), that is,
  \(\plII\win \schStrongSelGame{\mc D_{C_p(X)}}{\Omega_{C_p(X),\mathbf 0}}\).
\end{enumerate}
\end{theorem}

\begin{proof}
  (a) \(\Leftrightarrow\) (b) is a well-known result of Galvin.

  (b) \(\Leftrightarrow\) (c) is 4.3 of [Telgarksy 1975].
  
  (d) \(\Leftrightarrow\) (c) is clear, but we need to show that (a) \(\Leftrightarrow\) (d).
  So assume \(X\) is \(R^+\), which is equivalent to \(\Omega R^+\). 
  Let \(\sigma\) be a winning strategy for \(\plII\) in 
  \(\schStrongSelGame{\Omega_X}{\Omega_X}\). Let \(T(X)\) be the non-empty
  open sets of \(X\), and let \(s\in T(X)^{<\omega}\).
  Assume \(\tau(t)\in[X]^{<\omega}\) is defined for all \(t<s\),
  and \(\mc U_t\in\Omega_X\) is defined for all 
  \(\emptyset<t\leq s\). 

  Suppose that for all \(F\in[X]^{<\omega}\), there existed \(U_F\in T(X)\)
  containing \(F\) such that for all \(\mc U\in\Omega_X\),
  \(U_F\not=\sigma(\<\mc U_{s\rest 1},\dots,U_s,\mc U\>)\).
  Let \(\mc U=\{U_F:F\in[X]^{<\omega}\}\in\Omega_X\).
  Then \(\sigma(\<\mc U_{s\rest 1},\dots,U_s,\mc U\>)\)
  must equal some \(U_F\), demonstrating a contradiction.

  So there exists \(\tau(s)\in[X]^{<\omega}\) such that for all \(U\in T(X)\)
  containing \(\tau(s)\),
  there exists \(\mc U_{s\concat\<U\>}\in\Omega_X\) such that
  \(U=\sigma(\<\mc U_{s\rest 1},\dots,\mc U_s,\mc U_{s\concat\<U\>}\>)\).
  (To complete the induction, \(\mc U_{s\concat\<U\>}\) may be chosen
  arbitrarily for all other \(U\in T(X)\).)

  So \(\tau\) is a strategy for \(\plI\) in \(\Omega FO(X)\).
  Let \(\nu\) legally attack \(\tau\), so
  \(\tau(\nu\rest n)\subseteq \nu(n)\) for all \(n<\omega\).
  It follows that 
  \(\nu(n)=\sigma(\<\mc U_{\nu\rest 1},\dots,\mc U_{\nu\rest n},\mc U_{\nu\rest n+1}\>)\).
  Since \(\<\mc U_{\nu\rest 1},\mc U_{n\rest 2},\dots\>\) is a legal attack
  against \(\sigma\), it follows that
  \(\{\sigma(\<\mc U_{\nu\rest 1},\dots,\mc U_{\nu\rest n+1}\>):n<\omega\}=\{\nu(n):n<\omega\}\)
  is an \(\omega\)-cover. Therefore \(\tau\) is a winning strategy, 
  verifying \(\plI\win\Omega FO(X)\).

  The equivalence of (b), (e), (f), and (g) are given as 3.8 of [Tkachuk 2017].

  The equivalence of (h), (i), and (j) are due to Clontz.

  (j) \(\Leftrightarrow\) (f) follows from 3.18a of [Tkachuk 2017].
\end{proof}

Tkachuk showed the following in [CLOSEDDISCRETESELECTIONS].

\begin{theorem}
The following are equivalent for \(T_{3.5}\) spaces \(X\).
\begin{enumerate}[a)]
\item \(X\) is uncountable.
\item \(C_p(X)\) has discrete selectivity, that is,
      \(\plI\notprewin CD(C_p(X))\).
\end{enumerate}
\end{theorem}

Clontz came across these in grad school (didn't make it into the dissertation):

\renewcommand{\rothGame}[1]{\ensuremath{G_1(\mc O_{#1},\mc O_{#1})}}
\begin{theorem}
\(\plI\prewin PO(X)\) if and only if \(\plII\markwin\rothGame{X}\).
\end{theorem} 
\begin{proof}
Let \(\sigma\) be a winning Markov strategy for \(\plII\) in \(\rothGame{X}\).
Let \(n<\omega\). Suppose that for each \(x\in X\), there was an open neighborhood 
\(U_x\) of \(x\) where for every open cover \(\mc U\), \(\sigma(\mc U,n)\not=U_x\). 
Then \(\sigma(\{U_x : x\in X\},n)\not\in\{U_x:x\in X\}\), a contradiction.

So for each \(n<\omega\), there exists \(\tau(n)\in X\) such that for any
open neighborhood \(U\) of \(\tau(n)\), there exists an open cover \(\mc U_n\)
such that \(\sigma(\mc U_n,n)=U\). Then \(\tau\) is a predetermined strategy for
\(\plI\) in \(PO(X)\).

It is also winning: for every attack \(f\) against \(\tau\), note that
\(f(n)\) is an open neighborhood of \(\tau(n)\), so choose \(\mc U_n\)
such that \(\sigma(\mc U_n,n)=f(n)\). Then since \(\<\mc U_0,\mc U_1,\dots\>\)
is a legal attack against \(\sigma\), it follows that \(\{f(n):n<\omega\}\)
is an open cover of \(X\). Therefore \(\tau\) is a winning predetermined
strategy.

Now let \(\sigma\) be a winning predetermined strategy for \(\plI\) in \(PO(X)\).
For an open cover \(\mc U\) of \(X\) and \(n<\omega\), let \(\tau(\mc U,n)\)
be any open set in \(\mc U\) containing \(\sigma(n)\). It follows
that \(\tau\) is a winning Markov strategy for \(\plII\) in \(\rothGame{X}\).
\end{proof}

\begin{theorem}
\(\plII\markwin PO(X)\) if and only if \(\plI\prewin\rothGame{X}\).
\end{theorem}
\begin{proof}
Let \(\sigma\) be a winning predetermined strategy for \(\plI\) in \(\rothGame{X}\).
For \(x\in X\) and \(n<\omega\), let \(\tau(x,n)\) be any open set in \(\sigma(n)\)
containing \(x\). It follows that \(\tau\) is a winning Markov strategy for
\(\plII\) in \(PO(X)\).

Now let \(\sigma\) be a winning Markov strategy for \(\plII\) in \(PO(X)\).
We may defined the open cover \(\tau(n)=\{\sigma(x,n):x\in X\}\) of \(X\).
It follows that \(\tau\) is a winning predetermined strategy for \(\plI\)
in \(\rothGame{X}\).
\end{proof}


Combining with several other results in the literature,
we may observe the following.

\begin{theorem}
The following are equivalent for \(T_{3.5}\) spaces \(X\).
\begin{enumerate}[a)]
\item \(X\) is countable.
\item \(X\) is \(R^{+mark}\).
\item \(\plI\prewin PO(X)\). 
\item \(\plI\prewin FO(X)\).
\item \(\plI\prewin\Omega FO(x)\).
\item \(C_p(X)\) is first-countable.
\item \(\plI\prewin \gruConGame{C_p(X)}{\mathbf 0}\).
\item \(\plI\prewin CL(C_p(X),\mathbf 0)\).
\item \(\plI\prewin CD(C_p(X))\).
\item \(X\) is \(\Omega R^{+mark}\).
\item \(C_p(X)\) is \(sCFT^{+mark}\).
\item \(C_p(X)\) is \(sCDFT^{+mark}\).
\end{enumerate}
\end{theorem}

\begin{proof}
(a) implies (c) is straight-forward. So let
\(\sigma\) be a predetermined strategy for \(\plI\) in \(PO(X)\).
If \(x\not\in\{\sigma(n):n<\omega\}\), let \(f(n)=X\setminus\{x\}\)
for all \(n<\omega\). It follows that \(f\) is a legal
counter-attack for \(\plII\) defeating \(\sigma\). Thus not (a) implies
not (c).

The equivalence of (b) and (c) was shown above.

Clearly (c) implies (d), so we will see that (d) implies (a).
Let \(\sigma(n)\) be a pre-determined strategy for I for \(FO(X)\).
Towards a contradiction, suppose that there is some \(x \in X \smallsetminus \bigcup_n \sigma(n)\).
II could then play \(FO(X)\) as follows.
At round \(n\) II can play an open set \(U_n\) which contains \(\sigma(n)\) but excludes \(x\).
Then \(x \notin \bigcup_n U_n\), and so I has lost.
This is a contradiction.
So \(X = \bigcup_n \sigma(n)\), which means it is countable.

It also clear that (e) implies (d), we will show that (a) implies (e).
If \(X\) is countable, then so is \([X]^{<\omega}\), enumerate it as \(\{s_n : n \in \omega\}\).
I's pre-determined strategy for \(\Omega FO(X)\) is to play \(s_n\) are round \(n\).
Clearly whatever II plays will be an \(\omega\)-cover.
Thus (a) - (e) are equivalent.

It is well-known and easy to see that (a) is equivalent to (f).

To see that (f) implies (g), note that we can find a sequence of open sets \(U_n\) so that \(\mathbf 0 \in U_{n+1} \subseteq \overline{U_{n+1}} \subseteq U_n\) for all \(n\).
I simply plays \(U_n\) are turn \(n\), and whatever \(x_n\) are played by II must converge to \(x\).

Clearly (g) implies (h) which in turn implies (i), which is
equivalent to (a) by [CLOSEDDISCRETESELECTIONS] 
%Let \(\sigma(n)\) be a pre-determined strategy for I for \(CD(C_p(X))\).
%Towards a contradiction suppose \(X\) is uncountable.
%Without loss of generality we can assume that \(\sigma(n)\) is a basic open set, \([f_n,F_n,\varepsilon_n]\).
%If I is not already playing basic open sets, we can choose basic open subsets of I's play, and the resulting game will still produce a a sequence \(\{x_n : n \in \omega\}\) which is not closed discrete.
%Let \(F = \bigcup_n F_n\).
%As \(X\) is uncountable, we can choose a point \(x \in X \smallsetminus F\).
%Then II can play against \(\sigma\) by playing a function \(g_n\) at round \(n\) with the properties that \(g_n \in [f_n,F_n,\varepsilon_n]\) and \(g_n(x) = n\).
%The collection \(\{g_n : n \in \omega\}\) is then closed discrete, and so I has lost.
%This is a contradiction, so \(X\) must be countable.
%Thus (a) - (i) are equivalent.

Clontz showed that (j), (k), and (l) are equivalent.

Note that (j) implies (b) implies (a).
So the last thing we need to show is that (e) implies (j).
Let \(\sigma(n)\) be a pre-determined strategy for I for \(\Omega FO(X)\).
We define a Markov strategy, \(\tau(\mathcal{U},n)\) for II for \(\Omega R\) as follows.
At round \(n\) suppose I has played \(\mathcal{U}\).
As \(\mathcal{U}\) must be an \(\omega\)-cover, there is a \(U \in \mathcal{U}\) so that \(\sigma(n) \subseteq U\).
II plays such a \(U_n\).
Now suppose this game has been played according to \(\tau\), and that I has played \(\mathcal{U}_n\) for \(n < \omega\).
Then the sequence of open sets \(\tau(\mathcal{U}_n,n)\) forms a legal play against \(\sigma\) for \(\Omega FO(X)\).
Thus \(\{\tau(\mathcal{U}_n,n) : n \in \omega\}\) is an \(\omega\)-cover of \(X\) and so \(\tau\) is a winning Markov strategy.
\end{proof}

In that paper, Tkachuk characterizes \(\plII\win\Omega FO(X)\)
as the second player having an ``almost winning strategy''
(\(\plII\) can prevent \(\plI\) from constructing an \(\omega\)-cover
but perhaps not an arbitrary open cover)
in \(PO(X)\), which he conflates with \(FO(X)\) as they are
equivalent for ``completely'' winning perfect information strategies. 

But they cannot be interchanged in general.
Note that \(\plII\tactwin\Omega PO(2)\), where \(2\) is the two-point discrete space:
let \(\sigma(\<x\>)=\{x\}\). Since every \(\omega\)-cover of \(2\) includes \(2\),
and \(\sigma\) never produces \(2\), this is a winning tactic. But since \(2\)
is countable, \(2\) is \(\Omega R^{+mark}\).
So \(\Omega PO(X)\) is a very different game than those described previously.


Now we turn our attention to the opponent.

\begin{theorem}
The following are equivalent for all spaces \(X\).
\begin{enumerate}[a)]
\item \(\plII\win FO(X)\)
\item \(\plII\win PO(X)\)
\item \(\plI\win \rothGame{X}\)
\item \(\plI\prewin \rothGame{X}\)
\item \(\plII\markwin PO(X)\)
\item \(\plII\markwin FO(X)\)
\end{enumerate}
In particular, these are all equivalent to \(X\) not being \(R\).
\end{theorem}
\begin{proof}
(a) \(\Leftrightarrow\) (b) is 4.4 of [Telgarksy 1975].

The duality of \(PO(X)\) and \(\rothGame{X}\) for both players
when considering perfect information is a well-known result of Galvin.
So (b) is equivalent to (c).

The equivalence of (c) and (d) is just a restatement of Pawlikowski's
result that the Rothberger selection principle is equivalent
to \(\plI\notwin \rothGame{X}\), since the Rothberger selection principle
is equivalent to \(\plI\notprewin\rothGame{X}\).

(d) and (e) were shown to be equivalent above.

Finally, (f) implies (e) is obvious. 
Let \(b:\omega^2\to\omega\) be a bijection.
Given a winning Markov strategy \(\sigma\) for \(\plII\) in \(PO(X)\),
define \(\tau(F_n,n)=\bigcup\{\sigma(x(i,n),b(i,n)):i<\omega\}\)
where \(F_n=\{x(i,n):i<\omega\}\) (this indexing will cause at least one 
point to be repeated infinitely often, but this won't be a problem). 
So given an attack \(\<F_0,F_1,\dots\>\) \
against \(\tau\), consider the attack \(g\) against \(\sigma\), 
where \(g(n)=x_{b^{\leftarrow}(n)}\). It follows that
\[
  X
    \not=
  \bigcup\{\sigma(g(n),n):n<\omega\}
    =
  \bigcup\{\sigma(x(i,n),b(i,n)):i,n<\omega\}
    =
  \bigcup\{\tau(F_n,n):n<\omega\}
\]
and therefore \(\tau\) is a winning Markov strategy for \(\plII\).
Thus (e) implies (f).
\end{proof}

\begin{theorem}
The following are equivalent for all spaces \(X\).
\begin{enumerate}[a)]
\item \(\plII\win \Omega FO(X)\)
\item \(\plI\win G_1(\Omega_X,\Omega_X)\)
\item \(\plI\prewin G_1(\Omega_X,\Omega_X)\)
\item \(\plII\markwin \Omega FO(X)\)
\end{enumerate}
In particular, these are all equivalent to \(X\) not being \(\Omega R\).
\end{theorem}
\begin{proof}
Let \(\sigma\) be a winning strategy for \(\plII\) in \(\Omega FO(X)\).
For \(s\in([X]^{<\omega})^{<\omega}\), let 
\(\mc U_s=\{\sigma(s\concat\<F\>:F\in[X]^{<\omega}\}\).
Then define the strategy \(\tau\) for \(\plI\) by 
\(\tau(s)=\sigma(\<\mc U_{s\rest 0},\dots,\mc U_s\>)\).
Then every attack \(f\) against \(\tau\) yields \(g\in([X]^{<\omega})^\omega\)
such that \(f(n)=\sigma(g\rest n+1)\). Thus 
\(\{f(n):n<\omega\}=\{\sigma(g\rest n+1):n<\omega\}\)
is not an \(\omega\)-cover, so \(\tau\) is a winning strategy,
verifying that (a) implies (b).

The equivalence of (b) and (c) is given by Thm2 of
[http://eudml.org/doc/212209].

Let \(\sigma\) be a winning predetermined strategy for \(\plI\)
in \(G_1(\Omega_x,\Omega_x)\). For \(F\in[X]^{<\omega}\) and
\(n<\omega\), let \(\tau(F,n)\) be any open set in \(\sigma(n)\)
containing \(F\). It follows that \(\tau\) is a winning Markov
strategy for \(\plII\) in \(\Omega FO(X)\), verifying that
(c) implies (d).

(d) implies (a) is trivial, so the proof is complete.
\end{proof}

\(\Omega R\) is equivalent to all finite powers being \(R\):
Thm3 of [http://eudml.org/doc/212209].
These notions cannot coincide: see Thm9 of 
[http://dx.doi.org/10.1016/j.topol.2013.07.022] 
for a consisent example of a \(R\) space whose square
is not \(R\), so therefore not \(\Omega R\). 
Note the distinction with strategies for the opponent, as \(R^+\) is
equivalent to \(\Omega R^+\) and \(R^{+mark}\) is
equivalent to \(\Omega R^{+mark}\).

\begin{corollary}
The following are equivalent for all \(T_{3.5}\) spaces.
\begin{enumerate}[a)]
\item \(X\) is not \(\Omega R\)
\item \(\plII\win \Omega FO(X)\)
\item \(\plII\markwin \Omega FO(X)\)
\item \(\plI\win G_1(\Omega_X,\Omega_X)\)
\item \(\plI\prewin G_1(\Omega_X,\Omega_X)\)
\item \(C_p(X)\) is not \(sCFT\), that is, 
  \(\plI\prewin \schStrongSelGame{\Omega_{C_p(X),\mathbf 0}}{\Omega_{C_p(X),\mathbf 0}}\).
\item \(C_p(X)\) is not \(sCDFT\), that is,
  \(\plI\prewin \schStrongSelGame{\mc D_{C_p(X)}}{\Omega_{C_p(X),\mathbf 0}}\).
\end{enumerate}
\end{corollary}
\begin{proof}
(a)-(e) were just shown.
The equivalence of (a),(f),(g) was shown by Clontz.
\end{proof}

\begin{theorem}
The following are equivalent for Frechet-Urysohn spaces.
\begin{enumerate}[a)]
\item \(\plII\win\gruConGame{X}{x}\)
\item \(\plII\markwin\gruConGame{X}{x}\)
\item \(\plI\win\schStrongSelGame{\Omega_{X,x}}{\Omega_{X,x}}\)
\item \(\plI\prewin\schStrongSelGame{\Omega_{X,x}}{\Omega_{X,x}}\) 
\end{enumerate}
\end{theorem}
\begin{proof}
If \(\plII\win\gruConGame{X}{x}\), then by
[https://doi.org/10.1016/0016-660X(78)90032-6]
there exist sets \(A_n\in\Omega_{X,x}\) for \(n<\omega\) such
that for all \(x_n\in A_n\), \(x_n\not\to x\). 
So let \(\sigma(U,n)\in A_n\cap U\).
It follows that \(\sigma\) is a winning Markov strategy
for \(\plII\) in \(\gruConGame{X}{x}\), so (a)
is equivalent to (b).

Furthermore, let \(\tau(n)=A_n\). Since
\(x_n\not\to x\) implies \(\{x_n:n<\omega\}\not\in\Omega_{X,x}\)
as \(X\) is F-U, it follows that
\(\tau\) is a winning predetermined strategy for \(\plI\)
in \(\schStrongSelGame{\Omega_{X,x}}{\Omega_{X,x}}\).
Thus (a) implies (d) implies (c).

So finally, let \(\sigma\) be a winning strategy for \(\plI\) in
\(\schStrongSelGame{\Omega_{X,x}}{\Omega_{X,x}}\). 
For \(s\concat\<U\>\in {T_{X,x}}^{<\omega}\setminus\{\emptyset\}\),
let \(\tau(s\concat\<U\>)\in\sigma(\<\tau(s\rest 1),\dots,\tau(s)\>)\cap U\)
be a strategy for \(\plII\) in \(\gruConGame{X}{x}\).
If \(f\) attacks \(\tau\), then
\(\<\tau(f\rest 1),\tau(f\rest 2),\dots\>\) attacks \(\sigma\),
and therefore \(\{\tau(f\rest n+1):n<\omega\}\not\in\Omega_{X,x}\).
It follows that \(\tau(f\rest n+1)\not\to x\), so \(\tau\) is a
winning strategy for \(\plII\), verifying that (c) implies (a).
\end{proof}

\begin{corollary}
The following are equivalent for all \(T_{3.5}\) \(\gamma\) spaces.
\begin{enumerate}[a)]
\item \(X\) is not \(\Omega R\)
\item \(\plII\win \Omega FO(X)\)
\item \(\plII\markwin \Omega FO(X)\)
\item \(\plI\win G_1(\Omega_X,\Omega_X)\)
\item \(\plI\prewin G_1(\Omega_X,\Omega_X)\)
\item \(C_p(X)\) is not \(sCFT\), that is, 
  \(\plI\prewin \schStrongSelGame{\Omega_{C_p(X),\mathbf 0}}{\Omega_{C_p(X),\mathbf 0}}\).
\item \(C_p(X)\) is not \(sCDFT\), that is,
  \(\plI\prewin \schStrongSelGame{\mc D_{C_p(X)}}{\Omega_{C_p(X),\mathbf 0}}\).
\item \(\plI\win\schStrongSelGame{\Omega_{C_p(X),\mathbf 0}}{\Omega_{C_p(X),\mathbf 0}}\)
\item \(\plII\win\gruConGame{C_p(X)}{\mathbf 0}\)
\item \(\plII\markwin\gruConGame{C_p(X)}{\mathbf 0}\)
\end{enumerate}
\end{corollary}
\begin{proof}
By [https://doi.org/10.1016/0166-8641(82)90065-7],
\(X\) being \(\gamma\) is equivalent to \(C_p(X)\) being F-U.
\end{proof}


  \bibliographystyle{plain}
  \bibliography{../bibliography}


\end{document}
